\documentclass[a4j,fleqn,10pt]{jarticle}
\usepackage{comment}
\usepackage{amsmath}
\usepackage{amssymb}
\usepackage{cancel}
\usepackage{mathdots}
\usepackage{fancyhdr}
\usepackage{color}
\usepackage{multirow}
\usepackage{bxpapersize}% 読み込みだけでOK
%\setlength{\hoffset}{-1in}
%\setlength{\voffset}{-1in}
%\setlength{\oddsidemargin}{10mm}
%\setlength{\topmargin}{20mm}
%\setlength{\textwidth}{190mm} % 210mm - 10mm - 10mm
%\setlength{\textheight}{257mm} % 297mm - 20mm - 20mm
%\addtolength{\topmargin}{-\headheight}
%\addtolength{\topmargin}{-\headsep}
%\addtolength{\textheight}{-\footskip}
\pagestyle{fancy}
\lhead{グラハム数の簡単な?解説 by @Hari1000Bom}
\rhead{\today}
\begin{document}
\title{グラハム数}
\section{グラハム数の定義}
\begin{eqnarray}
G(x)&=&3 \uparrow^ {x} 3 \\
グラハム数&=&G^{64}(4)=\underbrace {G(G( \cdots G(G(4)) \cdots ))} _{64回入れ子}=
\left .
\begin{array} {c}
%
3 \underbrace {\uparrow \uparrow \uparrow \uparrow \uparrow \cdots \uparrow  \uparrow  \uparrow  \uparrow \uparrow }3 \\
3 \underbrace {\uparrow \uparrow \uparrow \uparrow \cdots \uparrow  \uparrow  \uparrow \uparrow }3 \\
\vdots \\
3 \underbrace {\uparrow \uparrow \uparrow \cdots \uparrow  \uparrow \uparrow }3 \\
3 \underbrace {\uparrow \uparrow \cdots \uparrow \uparrow }3 \\
3 \uparrow \uparrow \uparrow \uparrow 3 
\end{array}
\right \}
64階層
\end{eqnarray}
1段階目は3と3の間に$\uparrow$が4つ。\\
2段階目は3と3の間に$\uparrow$が1段階目の数、つまり$3 \uparrow \uparrow \uparrow \uparrow 3$ 個、挟み込まれます。\\
3段階目は3と3の間に$\uparrow$が2段階目の数だけ挟み込まれます。\\
このように前段階の数だけ、3と3の間に$\uparrow$を挟み込む操作を繰り返します。\\
これを64段階繰り返した数がグラハム数です。
\section{クヌースの矢印表記}
加算の繰り返しが乗算になり、乗算の繰り返しがべき乗になるように、
矢印表記は、べき乗の繰り返し、その繰り返し、更にその繰り返しといった、べき乗の上位演算を表すための表記法の1つです。\\
べき乗以上の演算は右から計算していきます。\\
{\small
\begin{tabular} [c] {|c|c||c|l|l|}
\hline
演算レベル
&演算名&表記
&\multicolumn{1}{|c|}{演算の定義}
&演算内容
\tabularnewline
\hline \hline
1&加算& $m + n $ & $=\underbrace {(m+1)} _{mの次の数} +(n-1)$&
$=m \underbrace {+1 +1 \dotsb +1} _{+1をn回繰り返す} $ 
\tabularnewline \hline
2&乗算& $m \times n$& $=m+ (m \times (n-1))$ & 
 $=\underbrace {m+ \dotsb +m+ m} _{mがn個} $

\tabularnewline \hline
3&べき乗& 
$ \left .
\begin{array} {c} 
m^n \\ 
m \uparrow n
\end{array} 
\right .$& 
$=m\times (m \uparrow (n-1))$ & 
$=\underbrace {m \times  \dotsb \times m \times m} _{mがn個}  $
\tabularnewline \hline
4&テトレーション& 
$\left .
\begin{array} {c} 
m \uparrow \uparrow n  \\ 
m \uparrow^{2} n  \\
^nm
\end{array}
\right . $& 
$=m\uparrow (m \uparrow \uparrow (n-1))$ &
$=\underbrace {m \uparrow  \dotsb \uparrow m \uparrow m} _{mがn個} $
\tabularnewline \hline
5&ペンテーション& 
$ \left .
\begin{array} {c}
m \uparrow \uparrow \uparrow n \\
m \uparrow^{3} n  
\end{array}
\right . $& 
$=m\uparrow \uparrow (m \uparrow ^{3} (n-1))$ & 
$=\underbrace {m \uparrow \uparrow  \dotsb \uparrow \uparrow m \uparrow \uparrow m} _{mがn個} $
\tabularnewline \hline
6 &ヘキセーション& 
$$
$ \left .
\begin{array}{c}
m \uparrow \uparrow \uparrow \uparrow n \\
m \uparrow^{4} n
\end{array}
\right . $
&
$=m\uparrow ^{3} (m \uparrow ^{4} (n-1))   $
&
$=\underbrace {
m \uparrow ^{3}
\dotsb
\uparrow ^{3} m \uparrow ^{3} m} 
_{mがn個} $ 
\tabularnewline \hline
$\vdots $ &$\vdots $&$\vdots $&\multicolumn{1}{|c|}{$\vdots$}&\multicolumn{1}{|c|}{$\vdots$}
\tabularnewline \hline
\end{tabular}
}
\section{グラハム数の1段目を計算してみる}
%3^^4
まずは$3 \uparrow \uparrow 4$から。
\begin{eqnarray}
3 \uparrow \uparrow \textcolor{green}{\bf{4}}&=& \underbrace {3 \uparrow 3 \uparrow 3 \uparrow 3} _{\bf{3が\textcolor{green}{4}個}}= \underbrace {3 ^{3 ^ {3 ^3}}} _{\bf{\textcolor{green}{4}段 }} \\
 &=& 3 ^ {3 ^ {27}} \\
 &=& 3 ^ {7\ 625\ 597\ 484\ 987} \\
 &=& 10^{\log_{10}(3 ^ {7\ 625\ 597\ 484\ 987})} \\
&=& 10^{ 7\ 625\ 597\ 484\ 987 \times \log_{10}(3)} \\
&\fallingdotseq& 10 ^ {7\ 625\ 597\ 484\ 987 \times 0.477} \\
 &\fallingdotseq & 10 ^{ 3\ 638\ 334\ 640\ 024.10} \\
&\fallingdotseq&  10 ^{ 10^{12.56}} \\
&\fallingdotseq&  10 ^{ 10^{10^{1.10}}}
\end{eqnarray}
10の指数、つまり肩に乗っている数が桁数になるので、$3 \uparrow \uparrow 4$はおよそ3兆6千4百億桁の数という計算結果が出ました。\\
「観測可能な宇宙に含まれる全ての原子の数」である$10^{80}$を超えてますが、「観測可能な宇宙が取りうる全ての状態の数」と言われるプロマキシマ($10^{10^{245}}$)よりは小さいです。
\\
\\
%3^^5
次は$3 \uparrow \uparrow 5$。
\begin{eqnarray}
3 \uparrow \uparrow \textcolor{green}{\bf{5}}&=& \underbrace {3 \uparrow 3 \uparrow 3 \uparrow 3 \uparrow 3}_{\bf{3が\textcolor{green}{5}個}}
= \underbrace {3^{3 ^{3 ^ {3 ^3}}}} _{\bf{\textcolor{green}{5}段 }} \\
 &=& 3 ^{ 3 \uparrow \uparrow 4} \\
 &=& 3 ^ {3 ^ {7\ 625\ 597\ 484\ 987}} \\
&\fallingdotseq& 3 ^ {10 ^{ 3\ 638\ 334\ 640\ 024.10}} \\
&=&  10 ^{
{\log_{10}
(3^{10 ^{ 3\ 638\ 334\ 640\ 024.10}})
}} 
\\
&=&  10 ^{
10 ^{  3\ 638\ 334\ 640\ 024.10} \times \log_{10}(3)
} \\
&=&  10 ^{
10 ^{  3\ 638\ 334\ 640\ 024.10} \times 10^{\log_{10}(\log_{10}(3))}
} \\
&\fallingdotseq&  10 ^{
10 ^{  3\ 638\ 334\ 640\ 024.10} \times 10^{\log_{10}(0.477)}
} \\
&\fallingdotseq&  10 ^ {10^{  3\ 638\ 334\ 640\ 024.10} \times 10^{-0.32}} \\
&= & 10 ^ {10^{ 3\ 638\ 334\ 640\ 024.10 \  -0.32}} \\
 &= & 10 ^{10^{ 3\ 638\ 334\ 640\ 023.78 }}\\
 &\fallingdotseq & 10 ^{10^{10^{12.56}}}\\
&\fallingdotseq& 10^{10^{10^{10^{1.10}}}}
\end{eqnarray}
はい、$3 \uparrow \uparrow 5$は、あまりに指数がでか過ぎて、(10の約3兆6千4百億乗)桁という数になります。\\
$3 \uparrow \uparrow 6$で、「宇宙論で使われた最大の数」である$10^{10^{10^{10^{10^{1.1}}}}}$と同じくらいになります。
\\
\\
% 3^^^3
今度はトリトリ(tritri)、$3 \uparrow \uparrow \uparrow 3$を計算してみます。
\begin{eqnarray}
{\rm tritri}&=& 3 \uparrow \uparrow \uparrow \textcolor{green}{\bf{3}}
%= 3 \uparrow \uparrow \uparrow \uparrow  \textcolor{cyan}{\bf{2}} 
\\
&=& \underbrace {3 \uparrow \uparrow 3 \uparrow \uparrow 3} _{\bf{3が\textcolor{green}{3}個}} \\
 &=& 3 \uparrow \uparrow (3 \uparrow 3 \uparrow 3)
\begin{comment}
 =
\begin{array} {c}
\underbrace {
\left .
\begin{array} {cll}
\underbrace {3 \uparrow 3 \uparrow \cdots \uparrow 3 \uparrow 3}&(=3 \uparrow \uparrow 3 \uparrow \uparrow 3)
&(=3 \uparrow \uparrow \uparrow \textcolor{green}{\bf{3}}) \\
\underbrace {3 \uparrow 3 \uparrow 3}&(=3 \uparrow \uparrow 3)
&(=3 \uparrow \uparrow \uparrow 2) \\
3&(=3 \uparrow \uparrow 1)&(=3 \uparrow \uparrow \uparrow 1)
\end{array}
\right \} \textcolor{green}{\bf{3段}}
} \\
\textcolor{cyan}{\bf{2列}}
\end{array}
\end{comment}
\\
 &=&\underbrace {
\left . \begin {array}{c} 3 ^ {3 ^ {\iddots ^{ 3 ^3}}} \end{array} \right \}  \left . 3^{3^3} 段\right \} 3段
} _{\bf{\textcolor{green}{3}列}}
\\
 &=& \left . \begin {array}{c} 3 ^ {3 ^ {\iddots ^{ 3 ^3}}} \end{array} \right \}  3 ^{27}段 \\
 &=& \left . \begin {array}{c} 3 ^ {3 ^ {\iddots ^{ 3 ^3}}} \end{array} \right \}   7{,}625{,}597{,}484{,}987段 \\
\end{eqnarray}
7兆6千億段を超える、3の指数タワーができてしまいました。\\
「宇宙論で使われた最大の数」が10の指数タワー5段なので、巨大な数の枕詞である「天文学的」の範疇を凄まじく超えてしまっています。\\
ここまで指数タワーが積み上がると、普通の感覚では数そのものと、その桁数がほぼ同じくらいになります。\\
\\
最後にグラハム数の1段階目(grahalと呼ばれています)を計算してみます。
\begin{eqnarray}
{\rm grahal} 
&=& 3 \uparrow \uparrow \uparrow \uparrow \textcolor{cyan}{\bf{3}} 
%=3 \uparrow \uparrow \uparrow \uparrow \uparrow \bf {2}
\\
&=& \underbrace {3 \uparrow \uparrow \uparrow 3 \uparrow \uparrow \uparrow 3}_{3が\textcolor{cyan}{\bf{3}}個} \\
&=& 3 \uparrow \uparrow \uparrow {\rm tritri} \\
&=&
\begin{array}{c}
3 \uparrow \uparrow \uparrow \left (
\left .
\begin{array}{c}
3 ^ {3 ^ {\iddots ^{ 3 ^3}}} 
\end{array}
\right \} 
\left . 
3^{3^3}段 \right \} 3段
\right )
\end{array}
\\
&=& 
\underbrace{
\left .
\left .
\left .
\left .
\left .
3 ^ {3 ^ {\iddots ^{ 3 ^3}}} 
\right \}
3 ^ {3 ^ {\iddots ^{ 3 ^3}}} 
\right \}
\cdots
\right \}
3 ^ {3 ^ {\iddots ^{ 3 ^3}}}
\right \}
3^{3^3}
\right \}
3
} _{
\underbrace{3 ^ {3 ^ {\iddots ^{ 3 ^3}}}} _{\underbrace{3^{3^3}}_{3段}段}階層
}
\end{eqnarray}
\newpage
\section{おまけ:テトレーションを加算のみで表現してみる}
\begin{eqnarray}
% 4*3
3 \times 4&=&
\left .
\begin{array} {c}
\underbrace {3+3+3+3} \\
4
\end{array}
\right .
\\
% 3^4
3 ^ 4&=&
\left .
\begin{array} {c}
\underbrace {3 \times 3 \times 3 \times 3} \\
4
\end{array}
\right .
=
\left .
\begin{array} {cll}
\underbrace {3+3+3+ \dotsb +3+3+3}&(=3 \times 3^3)&(=3^4=81)\\
\underbrace {3+3+ \dotsb +3+3}&(=3 \times 3^2)&(=3^3=27)\\
\underbrace {3+3+3} &(=3 \times 3)&(=3^2=9)\\
3&&(=3^1)
\end{array}
\right \} 
4
\end{eqnarray}
\\
% 3^^3
\footnotesize{
\begin{eqnarray}
3 \uparrow \uparrow 3&=&
\left .
\begin{array} {c}
\underbrace {
3 \uparrow 3 \uparrow 3
}\\
3 
\end{array}
\right . 
=
\left .
\begin{array} {c}
\underbrace {
3^{3^3}
}\\
3 
\end{array}
\right . 
%
=
\left . 
\begin{array} {c}
\underbrace {3 \times 3 \times \dotsb \times 3 \times 3 } \\
\underbrace {3 \times 3 \times 3 } \\
3
\end{array}
\right \} 3
\\
%
\nonumber \\
&=&
\begin{array} {c}
\underbrace {
\left .
\begin{array} {cr}
\underbrace {3+3+3+3+ \dotsb +3+3+3+3}&(=3^{27}=7{,}625{,}597{,}484{,}987)\\
\underbrace {3+3+3+ \dotsb +3+3+3}&(=3^{26}=2{,}541{,}865{,}828{,}329)\\
\vdots \\
\underbrace {3+3+ \dotsb +3+3}&(=3^3=27)\\
\underbrace {3+3+3} &(=3^2=9)\\
3&(=3^1)
\end{array}
\right \}
\left .
\begin{array} {cr}
\underbrace {3+3+ \dotsb +3+3}&(=3^3=27)\\
\underbrace {3+3+3} &(=3^2=9)\\
3&(=3^1)
\end{array}
\right \} 
3
} \\
3
\end{array}
\end{eqnarray}
}
% 3^^4
\begin{eqnarray}
3 \uparrow \uparrow 4&=&
\left .
\begin{array} {c}
\underbrace {
3 \uparrow 3 \uparrow 3 \uparrow 3
}\\
4
\end{array}
\right . 
=
\left .
\begin{array} {c}
\underbrace {
3^{3^{3^3}}
}\\
4
\end{array}
\right . 
%
=
\left . 
\begin{array} {c}
\underbrace {3 \times 3 \times 3 \times \dotsb \times 3 \times 3 \times 3 } \\
\underbrace {3 \times 3 \times \dotsb \times 3 \times 3 } \\
\underbrace {3 \times 3 \times 3 } \\
3
\end{array}
\right \} 4
\\
\nonumber \\
%
&=&
\left .
\begin{array} {c}
\underbrace {
\left .
\begin{array} {c}
\underbrace {3+3+3+3+ \dotsb +3+3+3+3}
\\
\underbrace {3+3+3+ \dotsb +3+3+3}\\
\vdots \\
\underbrace {3+3+ \dotsb +3+3}\\
\underbrace {3+3+3} \\
3
\end{array}
\right \}
\left .
\begin{array} {c}
\underbrace {3+3+3+3+ \dotsb +3+3+3+3}
\\
\underbrace {3+3+3+ \dotsb +3+3+3}\\
\vdots \\
\underbrace {3+3+ \dotsb +3+3}\\
\underbrace {3+3+3} \\
3
\end{array}
\right \}
\left .
\begin{array} {c}
\underbrace {3+3+ \dotsb +3+3}
\\
\underbrace {3+3+3}\\
3
\end{array}
\right \} 
3
} \\
4
\end{array}
\right .
\end{eqnarray}
% 3^^^3
\scriptsize {
\begin{eqnarray}
3 \uparrow \uparrow \uparrow 3&=&
\left .
\begin{array} {c}
\underbrace {
3 \uparrow \uparrow 3 \uparrow \uparrow 3
}\\
3
\end{array}
\right . 
%
=
\left .
\begin{array} {c}
\underbrace {
3 \uparrow 3 \uparrow \cdots \uparrow 3 \uparrow 3
} \\
\underbrace {
3 \uparrow 3 \uparrow 3
} \\
3
\end{array}
\right .
=
\left . 
\left .
\begin{array} {c}
\underbrace {3 \times 3 \times 3 \times \dotsb \times 3 \times 3 \times 3 } \\
\vdots \\
\underbrace {3 \times 3 \times \dotsb \times 3 \times 3 } \\
\underbrace {3 \times 3 \times 3 } \\
3
\end{array}
\right \}
\begin{array} {c}
\underbrace {3 \times 3 \times \dotsb \times 3 \times 3 } \\
\underbrace {3 \times 3 \times 3 } \\
3
\end{array}
\right \} 3
\\
\nonumber \\
%
&=&
\left .
\begin{array} {rc}
%3 \uparrow \uparrow 3 \uparrow \uparrow 3= 3 \uparrow \uparrow 7625597484987 =
&
\underbrace {
\left .
\left .
\begin{array} {c}
\underbrace {3+3+3+3+ \dotsb +3+3+3+3} \\
\underbrace {3+3+3+ \dotsb +3+3+3}\\
\vdots \\
\underbrace {3+3+ \dotsb +3+3}\\
\underbrace {3+3+3} \\
3
\end{array}
\right \}
\ \cdots \ 
\right \}
\left .
\begin{array} {c}
\underbrace {3+3+ \dotsb +3+3} \\
\underbrace {3+3+3}\\
3
\end{array}
\right \} 
3
} \\
%3 \uparrow \uparrow 3 =3^{3^3}=3^{27}=7625597484987 =
&
\underbrace {
\left .
\begin{array} {c}
\underbrace {3+3+3+3+ \dotsb +3+3+3+3}
\\
\underbrace {3+3+3+ \dotsb +3+3+3}\\
\vdots \\
\underbrace {3+3+ \dotsb +3+3}\\
\underbrace {3+3+3} \\
3
\end{array}
\right \}
\left .
\begin{array} {c}
\underbrace {3+3+ \dotsb +3+3} \\
\underbrace {3+3+3}\\
3
\end{array}
\right \} 
3
} \\
&3
\end{array}
\right .
\end{eqnarray}
}
\end{document}

