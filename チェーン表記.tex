\documentclass[a4j,fleqn]{jarticle}
\usepackage{comment}
\usepackage{amsmath}
\usepackage{amssymb}
\usepackage{cancel}
\usepackage{mathdots}
\usepackage{graphicx}
\setlength{\hoffset}{-1in}
\setlength{\voffset}{-1in}
\setlength{\oddsidemargin}{15mm}
\setlength{\topmargin}{20mm}
\setlength{\textwidth}{180mm} % 210mm - 10mm - 10mm
\setlength{\textheight}{267mm} % 297mm - 20mm - 20mm
\addtolength{\topmargin}{-\headheight}
\addtolength{\topmargin}{-\headsep}
\addtolength{\textheight}{-\footskip}
\usepackage{fancyhdr}
\usepackage{color}
\pagestyle{fancy}
\lhead{チェーン表記の解説 by @Hari1000Bom}
\rhead{\today}
\begin{document}
\section{チェーン表記とは}
チェーン表記は、矢印表記を拡張した表記法です。\\
ここに、矢印表記とチェーン表記の比較をしてみます。\\
\\
\begin{tabular} [c] {|c|c|}
\hline
矢印表記&チェーン表記\\
\hline \hline
$a \uparrow^{n} b$&$a \to b \to n$\\
\hline
$a ^ b=a \uparrow b$&$a \to b $\\
\hline
\end{tabular}
\\
\\
矢印表記では、2つの数の間に$\uparrow$が挟まっていましたが、チェーン表記では一番後ろに$\uparrow$の数が来るようになっています。
これが後々大きな意味を持ちます。
\section{チェーン表記の計算ルール}
矢印表記を展開しながら、同じ内容をチェーン表記してみることで、チェーン表記の計算ルールを探っていきます。\\
\\
\begin{tabular} [c] {|c|c|}
\hline
矢印表記&チェーン表記\\
\hline \hline
$3 \uparrow \uparrow 4$&$3 \to 4 \to 2$\\
$=3 \uparrow (3 \uparrow \uparrow 3)$&$=3 \to (3 \to 3 \to 2) \to 1$\\
$=3 \uparrow (3 \uparrow (3 \uparrow \uparrow 2))$&$=3 \to (3 \to (3 \to 2 \to 2) \to 1) \to 1$\\
$=3 \uparrow (3 \uparrow (3 \uparrow (3 \uparrow \uparrow 1)))$&$=3 \to (3 \to (3 \to (3 \to 1 \to 2) \to 1) \to 1) \to 1$\\
$=3 \uparrow (3 \uparrow (3 \uparrow 3 )))$&$=3 \to (3 \to (3 \to 3 \to 1) \to 1) \to 1$\\
&$=3 \to (3 \to (3 \to 3))$\\
\hline
\end{tabular}
\\
\\
ここから3つのルールが読み取れます。
\begin{enumerate}
 \item (2つ組を超える長さの)チェーン表記は、「チェーンの最後の数を1減らし、チェーンの最後から2番目の数を、「元のチェーンの、後ろから2番目の数を1減らしたもの」に置き換えたもの」に展開することができる。
 \item チェーンの最後の1は、$\to$ごと省略できる。
 \item チェーンの途中に1が出たら、その1を含めてそれ以降を省略できる。
\end{enumerate}
\section{計算例(1)}
\begin{eqnarray} 
3 \to 3 \to 3
&=&3 \uparrow ^{3} 3\\
=3 \to (3 \to 2 \to 3) \to 2 
&=& 3 \uparrow \uparrow (3 \uparrow^{3} 2)\\
=3 \to (3 \to (3 \ \ \bcancel{ \to 1 \to 3}) \to 2) \to 2
& =& 3 \uparrow \uparrow (3 \uparrow \uparrow (3 \bcancel{\uparrow^{3} 1}))\\
=3 \to (3 \to 3 \to 2) \to 2
& =& 3 \uparrow \uparrow (3 \uparrow \uparrow 3) \\
=3 \to (3 \to  (3 \to 2 \to 2) \to 1) \to 2 
& =& 3 \uparrow \uparrow (3 \uparrow (3 \uparrow \uparrow 2)) \\
=3 \to (3 \to  (3 \to  (3 \ \ \bcancel{ \to 1 \to 2}) \to 1) \to 1) \to 2
& =& 3 \uparrow \uparrow (3 \uparrow (3 \uparrow (3 \bcancel{\uparrow \uparrow }1))) \\
=3 \to (3 \to  (3 \to  3 \ \  \bcancel{\to 1}) \ \ \bcancel{\to 1}) \to 2\\
=3 \to (3 \to  (3 \to  3)) \to 2
&=&  3 \uparrow \uparrow (3 \uparrow (3 \uparrow 3)) \\
=3 \to (3 \to 27) \to 2
&=&3 \uparrow \uparrow (3 \uparrow 27)\\
=3 \to 7{,}625{,}597{,}484{,}987 \to 2
&=& 3 \uparrow \uparrow 7{,}625{,}597{,}484{,}987 \\
&=& \underbrace{3 \uparrow 3 \uparrow \dotsb \uparrow 3 \uparrow 3} _ 
{3が7{,}625{,}597{,}484{,}987個}
\end{eqnarray}
\section{4つ組のチェーン表記}
3つ組のチェーン表記は、矢印表記を書き直しただけに過ぎませんが、チェーン表記の特徴は、チェーンの長さが3つ組に限定されず、例えば、$3 \to 4 \to 2 \to 2$のように、伸ばせることです。\\
矢印表記では、4つ組のチェーンをそのまま書き表すことができませんが、チェーン表記の計算ルールを適用して3つ組のチェーンに展開することで、矢印表記に書き換えることができます。\\
\section{計算例(2)}
\begin{eqnarray} 
3 \to 4 \to \textcolor{red}{2} \to 2
&=&
3 \to 4 \to (3 \to 4 \ \ \bcancel{\to 1 \to 2}) \bcancel{ \to 1} \\
&=&
3 \to 4 \to (3 \to 4 ) \\
&(=&
\left .
\begin{array} {c}
3 \underbrace { \uparrow \uparrow \dotsb \uparrow \uparrow } 4\\
3 \uparrow 4
\end{array}
\right \} \textcolor{red}{2段}
)\\
&=&
3 \to 4 \to 81\\
3 \to 4 \to \textcolor{red}{3} \to 2
&=&
3 \to 4 \to (3 \to 4 \to 2 \to 2) \to 1 \\
&=&
3 \to 4 \to (3 \to 4 \to (3 \to 4 \ \ \bcancel{\to 1 \to 2})  \bcancel{\to 1}) \bcancel{\to 1} \\
&=&
3 \to 4 \to (3 \to 4 \to (3 \to 4 ) ) \\
&=&
\left .
\begin{array} {c}
3 \underbrace { \uparrow \uparrow \dotsb \uparrow \uparrow } 4\\
3 \underbrace { \uparrow \uparrow \dotsb \uparrow \uparrow } 4\\
3 \uparrow 4
\end{array}
\right \} \textcolor{red}{3段}
\\
\nonumber \\
3 \to 3 \to 3 \to 3
&=&
3 \to 3 \to (3 \to 3 \to 2 \to 3) \to 2 \\
&=&3 \to 3 \to (3 \to 3 \to  (3 \to 3 \ \ \bcancel{\to 1 \to 3})  \to 2) \to 2 \\
&=&3 \to 3 \to (\textcolor{blue}{\bold{3 \to 3 \to  (\textcolor{red}{3 \to 3} )  \to 2}}) \to 2 \\
&=&
3 \to 3 \to 
\left (
\bold{
\color{blue}
\left .
\begin{array} {c}
\bold{ 3 \underbrace { \uparrow \uparrow \dotsb \uparrow \uparrow } 3 }\\
\bold {\vdots} \\
\bold {3 \underbrace { \uparrow \uparrow \dotsb \uparrow \uparrow } 3}\\
\bold {3 \uparrow 3
}
\end{array}
\right \}
\color{black}
\textcolor{red}{
3 \uparrow 3
}
}
\right )
\to 2
\\
&=&
\begin{array} {c}
\left .
\begin{array} {c}
3 \underbrace { \uparrow \uparrow \dotsb \uparrow \uparrow } 3\\
\vdots \\
3 \underbrace { \uparrow \uparrow \dotsb \uparrow \uparrow } 3\\
3 \uparrow 3
\end{array}
\right \}
\color{blue}
\left .
\begin{array} {c}
\bold{ 3 \underbrace { \uparrow \uparrow \dotsb \uparrow \uparrow } 3 }\\
\bold {\vdots} \\
\bold {3 \underbrace { \uparrow \uparrow \dotsb \uparrow \uparrow } 3}\\
\bold {3 \uparrow 3
}
\end{array}
\right \}
\color{black}
\textcolor{red}{
3 \uparrow 3段
}
\end{array}
\\
3 \to 4 \to \textcolor{red}{3} \to 3
&=&
3 \to 4 \to (3 \to 4 \to 2 \to 3) \to 2 \\
&=&
3 \to 4 \to (\bold{\textcolor{blue}{3 \to 4 \to  (\textcolor{red}{3 \to 4})  \to 2}}) \to 2 \\
&=&
3 \to 4 \to 
\left (
\color{blue}
\left .
\begin{array} {c}
\bold{3 \underbrace { \uparrow \uparrow \dotsb \uparrow \uparrow } 4}\\
\bold{\vdots} \\
\bold{3 \underbrace { \uparrow \uparrow \dotsb \uparrow \uparrow } 4}\\
\bold{3 \uparrow 4}
\end{array}
\right \}
\color{red}
\bold {3 \uparrow 4}
\color{black}
\right )
\to 2
\\
&=&
\begin{array} {c}
\underbrace {
\left .
\begin{array} {c}
3 \underbrace { \uparrow \uparrow \dotsb \uparrow \uparrow } 4\\
\vdots \\
3 \underbrace { \uparrow \uparrow \dotsb \uparrow \uparrow } 4\\
3 \uparrow 4
\end{array}
\right \}
\color{blue}
\left .
\begin{array} {c}
\bold{3 \underbrace { \uparrow \uparrow \dotsb \uparrow \uparrow } 4}\\
\bold{\vdots} \\
\bold{3 \underbrace { \uparrow \uparrow \dotsb \uparrow \uparrow } 4}\\
\bold{3 \uparrow 4}
\end{array}
\right \}
\color{red}
\bold {3 \uparrow 4段}
\color{black}
} \\
\textcolor{red}{3列}
\end{array}
\end{eqnarray} 
% 3→4→3→4
\begin{eqnarray}
3 \to 4 \to \textcolor{red}{3} \to 4
&=&
3 \to 4 \to (3 \to 4 \to  ({\bf \textcolor{red}{3 \to 4}})  \to 3) \to 3 \\
&=&
3 \to 4 \to 
\left (
\begin{array} {c}
\color{blue}
\bold {
\underbrace {
\left .
\begin{array} {c}
{\bf 3 \underbrace { \uparrow \uparrow \dotsb \uparrow \uparrow } 4}\\
{\bf \vdots} \\
{\bf 3 \underbrace { \uparrow \uparrow \dotsb \uparrow \uparrow } 4}\\
{\bf 3 \uparrow 4}
\end{array}
\right \}
{\bf \cdots}
\left .
\begin{array} {c}
{\bf 3 \underbrace { \uparrow \uparrow \dotsb \uparrow \uparrow } 4}\\
{\bf \vdots} \\
{\bf 3 \underbrace { \uparrow \uparrow \dotsb \uparrow \uparrow } 4}\\
{\bf 3 \uparrow 4}
\end{array}
\right \}
\left .
\begin{array} {c}
{\bf 3 \underbrace { \uparrow \uparrow \dotsb \uparrow \uparrow } 4}\\
{\bf \vdots} \\
{\bf 3 \underbrace { \uparrow \uparrow \dotsb \uparrow \uparrow } 4}\\
{\bf 3 \uparrow 4}
\end{array}
\right \}
{\bf  3 \uparrow 4}
}
} \\
{\bf \textcolor{red}{3 \uparrow 4}}
\end{array}
\right )
\to 3
\\
&=&
\left .
\begin{array} {c}
\underbrace {
\left .
\begin{array} {c}
3 \underbrace { \uparrow \uparrow \dotsb \uparrow \uparrow } 4\\
\vdots \\
3 \underbrace { \uparrow \uparrow \dotsb \uparrow \uparrow } 4\\
3 \uparrow 4
\end{array}
\right \}
\cdots
\left .
\begin{array} {c}
3 \underbrace { \uparrow \uparrow \dotsb \uparrow \uparrow } 4 \\
\vdots \\
3 \underbrace { \uparrow \uparrow \dotsb \uparrow \uparrow } 4 \\
3 \uparrow 4
\end{array}
\right \}
\left .
\begin{array} {c}
3 \underbrace { \uparrow \uparrow \dotsb \uparrow \uparrow } 4 \\
\vdots \\
3 \underbrace { \uparrow \uparrow \dotsb \uparrow \uparrow } 4 \\
3 \uparrow 4
\end{array}
\right \}
3 \uparrow 4
} \\
\color{blue}
{\bf 
\underbrace {
\left .
\begin{array} {c}
{\bf 3 \underbrace { \uparrow \uparrow \dotsb \uparrow \uparrow } 4}\\
{\bf \vdots} \\
{\bf 3 \underbrace { \uparrow \uparrow \dotsb \uparrow \uparrow } 4}\\
{\bf 3 \uparrow 4}
\end{array}
\right \}
{\bf \cdots}
\left .
\begin{array} {c}
{\bf 3 \underbrace { \uparrow \uparrow \dotsb \uparrow \uparrow } 4}\\
{\bf \vdots} \\
{\bf 3 \underbrace { \uparrow \uparrow \dotsb \uparrow \uparrow } 4}\\
{\bf 3 \uparrow 4}
\end{array}
\right \}
\left .
\begin{array} {c}
{\bf 3 \underbrace { \uparrow \uparrow \dotsb \uparrow \uparrow } 4}\\
{\bf \vdots} \\
{\bf 3 \underbrace { \uparrow \uparrow \dotsb \uparrow \uparrow } 4}\\
{\bf 3 \uparrow 4}
\end{array}
\right \}
{\bf 3 \uparrow 4}
}
} \\
\color{black}
{\bf \textcolor{red}{3 \uparrow 4}}
\end{array}
\right \} 
\textcolor{red}{3}
\end{eqnarray} 
\end{document}
